\def\lesssim{\mathrel{\hbox{\rlap{\hbox{\lower3pt\hbox{$\sim$}}}\hbox{\raise2pt\hbox{$<$}}}}}

\documentclass[margin,line]{res}

\usepackage[
a4paper,
margin=0.95in,
headsep=4pt, % separation between header rule and text
]{geometry}
\usepackage{fontspec}
\setmainfont{Roboto}

\newsectionwidth{1.1in}
%\oddsidemargin -.5in
%\evensidemargin -.5in
\textwidth=5.4in
\itemsep=0in
\parsep=0in
\linespread{1.05}

\newenvironment{list1}{
	\begin{list}{\ding{113}}{%
			\setlength{\itemsep}{0in}
			\setlength{\parsep}{0in} \setlength{\parskip}{0in}
			\setlength{\topsep}{0in} \setlength{\partopsep}{0in} 
			\setlength{\leftmargin}{0.17in}}}{\end{list}}
\newenvironment{list2}{
	\begin{list}{$\bullet$}{%
			\setlength{\itemsep}{0in}
			\setlength{\parsep}{0in} \setlength{\parskip}{0in}
			\setlength{\topsep}{0in} \setlength{\partopsep}{0in} 
			\setlength{\leftmargin}{0.2in}}}{\end{list}}


\begin{document}
	%\name{Rajika Kuruwita \hspace{2.9in} Born 5th October, 1992, Sri Lanka \vspace*{.1in}}
	\name{Dr. Rajika Kuruwita \hspace{8.3cm} Citizenship: Australian}
	
	\begin{resume}
 \vspace*{-0.1cm}
		\section{\sc Contact Information}
		\begin{tabular}{@{}p{2.4in}p{3in}}
			Heidelberg Institute for  & {\it Tel:}    +49 176 2675 1570 \\         
			Theoretical Studies & {\it E-mail:}  rajika.kuruwita@h-its.org\\ 
			Schloß-Wolfsbrunnenweg 35 &{\it Website:}  https://rajikalk.github.io/index.html\\   
			69118 Heidelberg, Germany  & ORCID: 0000-0002-9236-2919 \\     
		\end{tabular}

\vspace*{-0.1cm}

		\section{\sc Research Interests}
		Star formation, binary and multiple star systems, protoplanetary disks and planets in binary star systems, MHD simulations, and software development.

\vspace*{-0.1cm}
  
		\section{\sc Education}
		{\bf Australian National University}, Canberra, Australia \hfill {\bf February, 2015 - January, 2019}\\
		\vspace*{-.1in}
		\begin{list1}
			\item[] {\bf PhD}
			\begin{list2}
				\vspace*{.05in}
				\item Thesis Topic:  ``The formation, evolution, and survivability of discs around young binary stars'' 
				\item Primary Supervisor: Associate Professor Christoph Federrath
				\item Secondary Supervisor: Professor Michael Ireland
			\end{list2}
		\end{list1}
		{\bf Macquarie University}, Sydney, Australia \hfill {\bf February, 2010 - January, 2015}\\ 
		\vspace*{-.1in}
		\begin{list1}
			\item[] {\bf MRes. Physics and Astronomy}
			\begin{list2}
				\vspace*{.05in}
				\item Thesis Topic:  ``Fallback disks and the end of the common envelope phase'' 
				\item Primary Supervisor:  Professor Orsola De Marco
				\item Secondary Supervisor: Assistant Professor Jan Staff
			\end{list2}
			\vspace*{.05in}
			\item[] {\bf BSc. Astronomy and Astrophysics}
		\end{list1}

\vspace*{-0.1cm}

		\section{\sc Employment History}
            {\bf Heidelberg Institute for Theoretical Studies}, Heidelberg, Germany\\
		{\em Independent Postdoc Fellow} \hfill {\bf October, 2019 - Present}\\
		Research the formation of binary and multiple star systems via numerical simulations.\\
            \vspace{-1cm}\\
            
		{\bf University of Copenhagen}, Copenhagen, Denmark\\
		{\em Post-doctorate researcher (Marie Sklodowska-Curie Interactions Fellow)} \hfill        {\bf April, 2019 - August, 2022}\\
		Investigate protostellar multiplicity and binarity on disk evolution.\\
		\vspace{-1cm}\\
  
		{\bf Australian National University}, Canberra, Australia\\
		{\em Research Assistant} \hfill {\bf February, 2019 - April, 2019}\\
		Research the formation of binary star systems via simulations.\\
		{\em Outreach Assistant} \hfill {\bf December, 2015 - April, 2019}\\
		Organise and run outreach observing and site tours for the public, school, scout, and private groups, as well as design activities for the observatory visitor centre.
		
		{\bf Macquarie University}, Sydney, Australia\\
		{\em Laboratory Demonstrator} \hfill {\bf February, 2014 - January, 2015}\\
		Taught lab experiments for undergraduate students. This also involved marking lab books.\\
		{\em Observatory and Planetarium Supervisor} \hfill {\bf February, 2010 - January, 2015}\\
		Coordinated groups, created tours/presentations, operated observatory and planetarium.\\
		{\em Vacation Scholarship Researcher} \hfill {\bf December, 2012 - February, 2013}\\
		Simulated light curves to understand the influence of exoplanets on the asteroseismological pulsation spectrum of stars.\\
		{\em Vacation Scholarship Researcher} \hfill {\bf January, 2012 - February, 2012}\\
		Carried out research on nanowires using white light interferometry.\\

\vspace*{-0.5cm}
  
		\section{\sc Time Awarded}
		\begin{list1}
			\item[] {\bf Australian National University 2.3m Telescope}
			\begin{list2}
				\item {\bf PI}: Building a Census of Protoplanetary Disks in Binary Star Systems (20 nights over 3 years)
			\end{list2}
                \item[]{\bf LUMI Supercomputer}
                \begin{list2}
				\item {\bf CO-I}: Embedded Disks: 24000000 core hours over 12 months
                \end{list2}
                \item[]{\bf PRACE}
                \begin{list2}
				\item {\bf CO-I}: Embedded Disks (2021250113): 40000000 core hours over 12 months
                \end{list2}
		\end{list1}
  
		\vspace{-0.3cm}
  
		\section{\sc Selected Talks}
            {\bf European Astronomical Society ASM} \hfill {\bf July, 2024}\\
		Invited Review Talk \hfill Padua, Italy\\
            {\bf ESO Star \& planet formation seminar} \hfill {\bf September, 2023}\\
		Invited Talk \hfill Garching, Germany\\
		{\bf Anton Pannekoek Institute for Astronomy} \hfill {\bf April, 2022}\\
		Invited Talk \hfill Amsterdam, The Netherlands\\
		{\bf Distorted Astrophysical Discs}\hfill {\bf May, 2021}\\
		Contributed Talk \hfill Cambridge, UK\\
		{\bf Niels Bohr Institute} \hfill {\bf January, 2019}\\
		Invited Talk \hfill Copenhagen, Denmark\\
		{\bf Sutherland Astronomical Society Incorporated} \hfill {\bf September, 2018}\\
		Invited Talk \hfill Sydney, Australia\\
		{\bf Franco-Australian Astrobiology and Exoplanet School and Workshop} \hfill {\bf December, 2017}\\
		Contributed Talk \hfill Canberra, Australia\\
		{\bf Mt Stromlo Students Seminars} \hfill {\bf December, 2016}\\
		Contributed Talk (Awarded Best Theme Talk) \hfill Canberra, Australia\\
		{\bf Star Formation} \hfill {\bf August, 2016}\\
		Computational Astrophysics splinter session (Invited) \hfill Exeter, UK
		
		\section{\sc Awards and Honors}
		\begin{list2}
                \item 2023: Isobel Rojas Travel Award recipient (3000EUR)
                \item 2023: Hochschulwettbewerb (national college competition) winners. Received 10000EUR to create a communication project about `Our Universe'.
                \item 2022: Became the first HITS Independent Research Fellow (includes funding of 5000EUR per year)
			\item 2021: Kvinder i Fysik (Danish Women in Physics) Prize 2021 Nominee
			\item 2020: European Union INTERACTIONS Fellowship
			\item 2017: Joan Duffield Research Supplementary Scholarship
			\item 2015: Australian Postgraduate Award
			\item 2013: Macquarie University Research Training Scholarship
			\item 2012: Vacation Scholarship (Macquarie University)
			\item 2011: Vacation Scholarship (Macquarie University)
		\end{list2}
		
		\section{\sc Teaching}
		{\bf Computational astrophysics lecturing} \hfill {\bf November, 2019 - February 2021}\\
		Gave post-graduate level lectures on computational astrophysics reviewing hydrodynamics and modelling shock waves.\\
            {\bf Laboratory demonstrator} \hfill {\bf February, 2014 - January, 2015}\\
	    Taught lab experiments for undergraduate students in physics and astronomy. I also marked lab books.\\

        \vspace*{-0.7
        cm}

            \section{\sc Supervision}
            {\bf  Niels Bohr Institute masters students} \hfill {\bf August, 2021 - 2022}\\
		I co-supervised three master's students who worked on producing synthetic observations from my simulations and built a pipeline using machine learning to fit synthetic observations to real observations of young protostars.\\
		{\bf  Niels Bohr Institute bachelors projects} \hfill {\bf February-April, 2021, 2022}\\
		Supervised 5 bachelor student groups on projects including modelling exoplanet interiors, and n-body simulations of the solar system and stellar systems.\\
        {\bf Mt Stromlo Observatory summer research} \hfill {\bf December, 2017 - February, 2018}\\
		Co-supervised honours student Isabella Gerard on a research project on turbulent magnetic fields and star formation. I am a co-author of the paper published from this project.\\
		{\bf Mt Stromlo Observatory winter school} \hfill {\bf June-July, 2017}\\
		Advised undergraduate students Lara Cullinane, Patrick Armstrong, Joshua Ho and Lillian Guo in planning observations and writing telescope proposals.
  
		\section{\sc Computer Skills} 
		\begin{list2}
			\item Computing Languages: Python, Fortran and HTML
			\item Applications: \LaTeX , \texttt{yt}, simulation codes \texttt{RAMSES}, \texttt{FLASH}, \texttt{DISPATCH} and Enzo, analysis of hdf5 files from hydrodynamic simulations, reducing observational data in fits files, retrieving radial velocities.  
			\item Operating Systems:  Unix/Linux, Windows, and Mac.
		\end{list2}
		
		\section{\sc Other Academic Services}
		\begin{list2}
			\item Reviewer for Monthly Notices of the Royal Astronomical Society
			\item Founded of Astronomy on Tap Copenhagen in 2020.
			\item Treasurer of Kvinder i Fysik (the Danish Women in Physics Society) from 2019 to 2022.
			\item Contributed popular science articles to the Sunday Space in the Canberra Times. 
			\item Member of the Local Organising Committee for the 2017 Harley Wood Winter School and Annual Scientific Meeting of the Astronomical Society of Australia.
			\item Member of the Science Organising Committee for the 2016 Harley Wood Winter School.
			\item Chair of the Organising Committee for the 2016 Mt Stromlo Student Seminars.
		\end{list2}
		
		%\vspace{0.3in}
		\section{\sc Referee Details}
		\begin{list2}
			%\item Professor Orsola De Marco, Department of Physics and Astronomy, Macquarie University, Sydney NSW 2109, Australia. tel: +61 2 9850 4241, email: orsola.demarco@mq.edu.au
			%\item Associate Professor Michael Ireland, Research School of Astronomy and Astrophysics, Australian National University, Research School of Astronomy \& Astrophysics, Mount Stromlo Observatory, Cotter Road, Weston Creek, ACT 2611, Australia. tel: +61 2 6125 0288, email: michael.ireland@anu.edu.au
			\item Associate Professor Troels Haugb{\o}lle, Center for Star and Planet Formation, University of Copenhagen, Geology Museum, Øster Voldgade 5-7, 1350 København K, Denmark. Tel: +45 35 32 11 41. Email: haugboel@nbi.ku.dk
			\item Associate Professor Christoph Federrath, Research School of Astronomy and Astrophysics, Australian National University, Mount Stromlo Observatory, Cotter Road, Weston Creek, ACT 2611, Australia. Tel: +61 2 6125 0217. Email: christoph.federrath@anu.edu.au
                \item Dr. Fabian Schneider, Heidelberg Institute for Theoretical Studies, Schloss-Wolfsbrunnenweg 35, 69118 Heidelberg, Germany. Tel: +49 6221 533 334. Email: fabian.schneider@h-its.org
			\item Professor Jes Kristian J{\o}rgensen, Center for Star and Planet Formation, University of Copenhagen, Geology Museum, Øster Voldgade 5-7, 1350 København K, Denmark. Tel: +45 35 32 41 86. Email: jeskj@nbi.ku.dk
                %\item Dr Brad Tucker, Research School of Astronomy and Astrophysics, Australian National University, Research School of Astronomy \& Astrophysics, Mount Stromlo Observatory, Cotter Road, Weston Creek, ACT 2611, tel: +61 2 6125 6711, email: brad.tucker@anu.edu.au 
		\end{list2}
	
		%\newpage
\section{\sc Refereed Publications}
\begin{list1}
        \item[]{\bf Kuruwita} et al, \emph{Protostellar spin-up and fast rotator formation through binary star formation}, 2024, \emph{Accepted at Astronomy \& Astrophysics}
	\begin{list2}
		\item Lead author, and conductor of research and analysis.\\
	\end{list2}

        \item[]{\bf Kuruwita} \& Haub{\o}lle, \emph{The contribution of core-fragmentation on protostellar multiplicity}, 2023, Astronomy \& Astrophysics, 674, A196
	\begin{list2}
		\item Lead author, and conductor of research and analysis.\\
	\end{list2}
	\item[]{\bf Kuruwita et al.}, \emph{The dependence of episodic accretion on eccentricity during the formation of binary stars}, 2020, Astronomy \& Astrophysics, 641, A59
	\begin{list2}
		\item Lead author, and conductor of research and analysis.\\
	\end{list2}
	\item[] {\bf Kuruwita \& Federrath}, \emph{The role of turbulence during the formation of circumbinary disks}, 2019, Monthly Notices of the Royal Astronomical Society, 486, 3647-3663
	\begin{list2}
		\item Lead author, and conductor of research and analysis.\\
	\end{list2}
	\item[] {\bf Kuruwita et al.}, \emph{Multiplicity of disc-bearing stars in Upper Scorpius and Upper Centaurus-Lupus}, 2018, Monthly Notices of the Royal Astronomical Society, 480, 5099–5112
	\begin{list2}
		\item Lead author, and conductor of research and analysis.
		\item Collected the majority of observations.\\
	\end{list2}
	\item[] {\bf Kuruwita et al.}, \emph{Binary star formation and the outflows from their discs}, 2017, Monthly Notices of the Royal Astronomical Society, 470, 1626-1641
	\begin{list2}
		\item Lead author, and conductor of research and analysis.\\
	\end{list2}
	\item[] {\bf Kuruwita et al.}, \emph{Considerations on the role of fall-back discs in the final stages of the common envelope binary interaction}, 2016, Monthly Notices of the Royal Astronomical Society, 461, 486-496
	\begin{list2}
		\item Lead author, and conductor of research and analysis.\\
	\end{list2}

 \item[]{Li, S. {\bf et al}}, \emph{Observations of high-order multiplicity in a high-mass stellar protocluster}, 2024, Nature Astronomy, https://doi.org/10.1038/s41550-023-02181-9
 \begin{list2}
\item Used my models from Kuruwita \& Haugb{\o}lle (2023) to interpret statistics of this massive star-forming region.\\
\end{list2}

\item[]{Tuhtan, V., Al-Belmpeisi, R., Christensen, M. B., {\bf Kuruwita, R.}, \& Haugb{\o}lle, T.}, \emph{Simulated Analogues I: apparent and physical evolution of young binary protostellar systems}, 2024, \emph{In review at MNRAS}
	\begin{list2}
		\item Co-supervised Vito, Rami, and Mikkel for their Master's thesis\\
	\end{list2}

\item[]{Al-Belmpeisi, R., Tuhtan, V., Christensen, M. B., {\bf Kuruwita, R.}, \& Haugb{\o}lle, T.}, \emph{Simulated analogues II: a new methodology for non-parametric matching
of models to observations}, 2024, \emph{Accepted at MNRAS}
	\begin{list2}
		\item Co-supervised Vito, Rami, and Mikkel for their Master's thesis\\
	\end{list2}

 \item[]{Evans, E. {\bf et al}}, \emph{Orbital Architectures of Planet-Hosting Binaries III. Testing Mutual Inclinations of Stellar and Planetary Orbits in Triple-
Star Systems}, 2024, \emph{in review at MNRAS}
 \begin{list2}
\item Took observations used in this paper.\\
\end{list2}

 
	\item[]{J{\o}rgensen, J. \& {\bf Kuruwita, R.} et al}, \emph{Binarity of a protostar affects the evolution of the disk and planets}, 2021, Nature, Volume 606, Issue 7913, p.272-275
	\begin{list2}
		\item Lead the theoretical component of paper. Conducted analysis of simulations used for comparison with observations.\\
	\end{list2}
	\item[] Gerrard, I., Federrath, C., \&  {\bf Kuruwita, R.}, \emph{The role of magnetic field structure in the launching of protostellar jets}, 2019, Monthly Notices of the Royal Astronomical Society, 485, 5532-5542
	\begin{list2}
		\item Co-supervised Gerrard in running simulations and analysing them\\
	\end{list2}
	\item[] Green {\bf et al.}, \emph{Testing the binary trigger hypothesis in FUors}, 2016, The Astrophysical Journal, 830, 29
	\begin{list2}
		\item Obtained observational data with Keck and contributed to paper writing.\\
	\end{list2}
	\item[] Childress {\bf et al.}, \emph{The ANU WiFeS SuperNovA Programme (AWSNAP)}, 2016, Publications of the Astronomical Society of Australia, 33, 29
	\begin{list2}
		\item Obtained observational data with Australian National University 2.3m telescope.\\
	\end{list2}
	\item[] Little {\bf et al.}, \emph{Phase-stepping interferometry of GaAs nanowires: Determining nano-wire radius}, 2013, Applied Physical Letters, 103, 161107
	\begin{list2}
		\item Obtained experimental data with white light interferometry of nanowires.\\
	\end{list2}
\end{list1}
		
		%\newpage
		%\section{\sc Statement of Research Interests and Goals}
		
		%The importance of binary stars has been emphasised regularly with a significant portion of stars ($\sim50\%$) being born in binary or multiple-star systems (Raghavan et al. (2010), Moe \& Stefano (2017)). Binary stars are also the progenitors to many phenomena such as Type 1a supernovae used for cosmology, gravitational waves and certain morphologies of planetary nebulae. Due to the significance that binaries have over various astrophysical processes, my research has mostly focused on the evolution of binary star systems.
		
		%My Master of Research focused on the fall-back of bound material onto post-common envelope binaries. Previous simulations of the common envelope phase of binary stars found that the giant envelope is lifted during this phase, but remains mostly bound to the system. At the same time, the orbital separation is greatly reduced, but in most simulations, it levels off at values larger than measured from observations (Passy et al. 2012). We conjectured that during the post-in-spiral phase, the bound envelope gas would return to the system and provide another opportunity to further unbind the envelope and tighten the orbit. This work was carried out using 3D hydrodynamical simulations with the code Enzo and continued from the work of Passy et al. (2012). We found that the simulated fall-back event reduces the orbital separation efficiently, but fails to unbind the gas before the separation levels off once again (Kuruwita et al. 2016). We also found that more massive fall-back disks reduce the orbital separation more efficiently, but the efficiency of unbinding remains invariably very low. From these results, we deduced that unless a further energy source contributes to unbinding the envelope, all common envelope interactions would result in mergers. On the other hand, additional energy sources are unlikely to help, on their own, to reduce the orbital separation.
		
		%With the advent of the Kepler space telescope and the increasing number of known exoplanets my interest pivoted towards planets around binary star systems. To date, we know of $\sim700$ planets in binary systems with $\sim20$ known planets in a circumbinary orbit. The majority of work on planet formation has focused on single stars. This is because humans have an anthropocentric bias towards single stars as our solar system has one star. However, with the increasing number of planets in binaries being discovered I wanted to investigate the environments in which planets in binaries could form. My PhD work focuses on binary star formation and the dynamical evolution of protoplanetary disks in these systems. I have been conducting both theoretical and observational work in these areas of research. 
		
		%The theoretical focus of my PhD uses 3D ideal magneto-hydrodynamical simulations with the code FLASH to simulate binary star formation. I investigate how binarity affects the accretion of disk material, outflows, radiative feedback and dynamical evolution of the disks. 
		
		%For my first project I simulated the formation of a single star, a tight binary star ($ a\sim$~AU) and a wide binary star ($a\sim45$~AU). We studied the outflows and jets from these systems to understand the contributions the circumstellar and circumbinary disks have on the efficiency and morphology of the outflows. In the single star and tight binary case, we obtained a single pair of jets launched from the system, while in the wide binary case, two pairs of jets were observed (Kuruwita et al. 2017). This implied that in the tight binary case, the contribution of the circumbinary disk on the outflow is greater than that in the wide binary case. We also found that the single-star case is the most efficient at transporting mass, and linear and angular momentum from the system, while the wide binary case is less efficient. By studying the magnetic field structure, we deduced that the outflows in the single star and tight binary star case are magnetocentrifugally driven, whereas, in the wide binary star case, the outflows are driven by a magnetic pressure gradient.
		
		%Further analysis of my wide binary simulations saw the disruption of circumstellar disks as the system passed periastron passage. This creates a hostile environment for planets to form. The next series of simulations investigate this disruption of circumstellar disks but also introduces turbulence to the molecular cores that produce our binary stars. This is a natural progression in simulations as the universe is a turbulent place and plays a significant role in the fragmentation and distribution of angular momentum. This work finds that the inclusion of turbulence helps to build larger circumbinary disks, while also assisting with the ordering of magnetic fields through the disk for efficient launching of outflows (Kuruwita et al. 2018, \emph{Submitted} available at https://arxiv.org/abs/1810.10375).
		
		%The observational work of my PhD was performed to find the multiplicity of young disk-bearing objects in Upper Scorpius and Upper Centaurus-Lupus to determine the influence of multiplicity on disk persistence after $\sim5-20$~Myr. Disks were identified using infra-red (IR) excess from the Wide-field Infrared Survey Explorer (WISE) survey. Our survey consisted of 55 US members and 28 UCL members, using spatial and kinematic information to assign a probability of membership (Rizzuto et al. 2015). Spectra were gathered from the ANU 2.3m telescope using the Wide Field Spectrograph (WiFeS) to detect radial velocity variations that indicate the presence of a companion. We identified 2 double-lined spectroscopic binaries, both of which have strong IR excess. We found the binary fraction of disk-bearing stars in US and UCL for periods up to 20 years to be $0.06^{+0.07}_{-0.02}$ and $0.13^{+0.06}_{-0.03}$ respectively (Kuruwita et al. 2018). Based on the multiplicity of field stars, we obtained an expected binary fraction of $0.12^{+0.02}_{-0.01}$. The determined binary fractions for disk-bearing stars do not vary significantly from the field, suggesting that the overall lifetime of disks may not differ between single and binary star systems. This may also suggest that the likelihood of planet formation around binary stars is similar to that of single stars.
		
		%At the American Museum of Natural History I would continue my numerical work on binary star formation and evolution. I am currently conducting research on episodic accretion during binary star formation and testing the binary trigger hypothesis. Episodic accretion caused by dynamical interactions has been investigated observationally (Green et al. (2016), Tofflemire et al. (2017)) and some theoretical work has been done with simple disk simulations (de Val-Borro et al, (2011), Gomez de Castro et al, (2013)). My simulations are some of the first showing episodic accretion correlated with the period of the formed binaries beginning at the collapse of the molecular core. I would like to investigate this episodic accretion on outflows and feedback during star formation. This work will also contribute to understanding radiation feedback during binary star formation. In the past year, I have been incorporating a radiation module (Buntemeyer et al. 2014) into FLASH to understand how binarity affects radiation feedback and disk evolution, and accretion luminosity will play a significant role.
		
		%I would also like to conduct research into determining the separation ranges in which circumbinary planets are likely to form. This will help constrain targeted observations searching for these types of planets. Dynamical interactions and the ejection of a third companion are believed to be a common pathway for creating short-period binaries (Armitage \& Clarke (1997), Reipurth (2000)). The ejection of this third companion may disturb circumbinary disks from which planets would form. Long-period binaries are also more likely to host circumstellar disks than circumbinary disks, therefore the formation of circumbinary planets would be reduced around wider binaries. I would like to conduct a suite of simulations to assist the search for circumbinary planets by determining the type of binaries that are likely to form such planets.
		
		%Also along the lines of short-period binaries forming via the ejection of a third companion, I would also like to investigate the survivability of disks during such dynamic interactions. For example, the systems AK Sco and DQ Tau are two very short period binary systems ($\sim13$ and $\sim16$~days respectively (G{\"u}nther \& Kley, 2002)). These systems also host circumbinary disks. If these systems formed via the ejection of a third companion did these circumbinary disks survive the ejection or form at later stages?
		
		%My research interests cover a wide range from the birth of binary stars, to the evolution of disks and the formation of planets in these systems. If successful in this application, I believe I have the skills and background to integrate into the research teams at the American Museum of Natural History. My experience with numerical simulations, especially magneto-hydrodynamical codes, will contribute to rapidly becoming an active and efficient team member.
		
	\end{resume}
	
\end{document}









\documentclass{article}
\usepackage[utf8]{inputenc}

\title{CV_long}
\author{Rajika Kuruwita}
\date{October 2022}

\begin{document}

\maketitle

\section{Introduction}

\end{document}
