\def\lesssim{\mathrel{\hbox{\rlap{\hbox{\lower3pt\hbox{$\sim$}}}\hbox{\raise2pt\hbox{$<$}}}}}

\documentclass[margin,line]{res}

\usepackage[
a4paper,
margin=0.95in,
headsep=4pt, % separation between header rule and text
]{geometry}

\newsectionwidth{1in}
%\oddsidemargin -.5in
%\evensidemargin -.5in
\textwidth=5.4in
\itemsep=0in
\parsep=0in
\linespread{1.05}

\newenvironment{list1}{
	\begin{list}{\ding{113}}{%
			\setlength{\itemsep}{0in}
			\setlength{\parsep}{0in} \setlength{\parskip}{0in}
			\setlength{\topsep}{0in} \setlength{\partopsep}{0in} 
			\setlength{\leftmargin}{0.17in}}}{\end{list}}
\newenvironment{list2}{
	\begin{list}{$\bullet$}{%
			\setlength{\itemsep}{0in}
			\setlength{\parsep}{0in} \setlength{\parskip}{0in}
			\setlength{\topsep}{0in} \setlength{\partopsep}{0in} 
			\setlength{\leftmargin}{0.2in}}}{\end{list}}


\begin{document}
	%\name{Rajika Kuruwita \hspace{2.9in} Born 5th October, 1992, Sri Lanka \vspace*{.1in}}
	\name{Dr. Rajika Kuruwita}
	
	\begin{resume}
		\section{\sc Contact Information}
		\vspace{.05in}
		\begin{tabular}{@{}p{2.4in}p{3in}}
			Centre for Star and Planet Formation  & {\it Tel:}    +61 02 9850 7111 \\         
			University of Copenhagen & {\it E-mail:}  rajikakuruwita@gmail.com\\ 
			Øster Voldgade 5-7 &{\it Website:}  https://rajikalk.github.io/index.html\\   
			DK-1350, Copenhagen, Denmark  & ORCID: 0000-0002-9236-2919 \\     
		\end{tabular}
		
		\section{\sc Research Interests}
		Star formation, binary and multiple star systems, protoplanetary disks and planets in binary star systems, MHD simulations, software development.
		
		\section{\sc Education}
		{\bf Australian National University}, Canberra, Australia \hfill {\bf February, 2015 - January, 2019}\\
		\vspace*{-.1in}
		\begin{list1}
			\item[] {\bf PhD}
			\begin{list2}
				\vspace*{.05in}
				\item Thesis Topic:  ``The formation, evolution, and survivability of discs around young binary stars'' 
				\item Primary Supervisor: Associate Christoph Federrath
				\item Secondary Supervisor: Associate Professor Michael Ireland
			\end{list2}
		\end{list1}
		{\bf Macquarie University}, Sydney, Australia \hfill {\bf February, 2010 - January, 2015}\\ 
		\vspace*{-.1in}
		\begin{list1}
			\item[] {\bf MRes. Physics and Astronomy}
			\begin{list2}
				\vspace*{.05in}
				\item Thesis Topic:  ``Fallback disks and the end of the common envelope phase'' 
				\item Primary Supervisor:  Professor Orsola De Marco
				\item Secondary Supervisor: Assistant Professor Jan Staff
			\end{list2}
			\vspace*{.05in}
			\item[] {\bf BSc. Astronomy and Astrophysics}
		\end{list1}
		
		\section{\sc Employment History}
		{\bf University of Copenhagen}, Copenhagen, Denmark\\
		{\em Post-doctorate researcher (European Union INTERACTIONS fellow)} \hfill {\bf April, 2019 - Present}\\
		Research the formation of binary and multiple star systems via numerical simulations.\\
		
		{\bf Australian National University}, Canberra, Australia\\
		{\em Research Assistant} \hfill {\bf February, 2019 - April, 2019}\\
		Research the formation of binary stars systems via simulations.\\
		{\em Outreach Assistant} \hfill {\bf December, 2015 - April, 2019}\\
		Organise and run outreach observing and site tours for the public, school, scout and private groups, as well as design activities for the observatory visitor centre.
		
		{\bf Macquarie University}, Sydney, Australia\\
		{\em Laboratory Demonstrator} \hfill {\bf February, 2014 - January, 2015}\\
		Taught lab experiments for undergraduate students. This also involved marking lab books.\\
		{\em Observatory and Planetarium Supervisor} \hfill {\bf February, 2010 - January, 2015}\\
		Coordinated groups, created tours and presentations, operated observatory and planetarium.\\
		{\em Vacation Scholarship Researcher} \hfill {\bf December, 2012 - February, 2013}\\
		Simulated light curves to understand the influence of exoplanets on the asteroseismological pulsation spectrum of stars.\\
		{\em Vacation Scholarship Researcher} \hfill {\bf January, 2012 - February, 2012}\\
		Carried out research on nanowires using white light interferometry.\\
		
		\section{\sc Telescope Time Awarded}
		\begin{list1}
			\item[] {\bf Australian National University 2.3m Telescope}
			\begin{list2}
				\item {\bf PI}: Building a Census of Protoplanetary Disks in Binary Star Systems (4 nights)
				\item {\bf PI}: Building a Census of Circumbinary Protoplanetary Disks (3 nights)
				\item {\bf PI}: Building a Census of Circumbinary Protoplanetary Disks (6 nights)
				\item {\bf PI}: Building a Census of Circumbinary Protoplanetary Disks (7 nights)
			\end{list2}
		\end{list1}
		\vspace{-0.5cm}
		\section{\sc Talks}
		{\bf Early Phases of Star Formation ()Focus Group)} \hfill {\bf April, 2022}\\
		Contributed Talk \hfill Ringberg, Germany\\
		{\bf Journal Club} \hfill {\bf April, 2022}\\
		Invited Talk \hfill Amsterdam, The Netherlands\\
		{\bf Astronomy on Tap} \hfill {\bf August, 2021}\\
		Invited Talk \hfill Copenhagen, Denmark\\
		{\bf StarPlan Science days}\hfill {\bf June, 2021}\\
		Contributed Talk \hfill Copenhagen, Denmark\\
		{\bf Distorted Astrophysical Discs}\hfill {\bf May, 2021}\\
		Contributed Talk \hfill Cambridge, UK\\
		{\bf Transient Tuesday} \hfill {\bf March, 2021}\\
		Invited Talk \hfill Copenhagen, Denmark\\
		{\bf ESO Hypatia Collouium}\hfill {\bf February, 2021}\\
		Contributed Talk \hfill Online\\
		{\bf Ramses User Meeting} \hfill {\bf September, 2019}\\
		Contributed Talk \hfill Copenhagen, Denmark\\
		{\bf Annual Danish Astronomy Meeting} \hfill {\bf May, 2019}\\
		Contributed Talk \hfill Nyborg, Denmark\\
		{\bf Niels Bohr Institute} \hfill {\bf January, 2019}\\
		Invited Talk \hfill Copenhagen, Denmark\\
		{\bf Sutherland Astronomical Society Incorporated} \hfill {\bf September, 2018}\\
		Invited Talk \hfill Sydney, Australia\\
		{\bf Greenlight for Girls National Science Week} \hfill {\bf August, 2018}\\
		Invited Talk \hfill Canberra, Australia\\
		{\bf University of T\"ubingen} \hfill {\bf May, 2018}\\
		Astronomy Seminar \hfill T\"ubingen, Germany\\
		{\bf Heidelberg Institute for Theoretical Astrophysics} \hfill {\bf May, 2018}\\
		Astronomy Seminar \hfill Heidelberg, Germany\\
		{\bf Max Planck Institute for Astronomy} \hfill {\bf May, 2018}\\
		Planet and Star Formation Seminar \hfill Heidelberg, Germany\\
		{\bf Hamburg Observatory} \hfill {\bf May, 2018}\\
		Astronomy Seminar \hfill Hamburg, Germany\\
		{\bf Annual Scientific Meeting of the Astronomical Society of Australia} \hfill {\bf June, 2018}\\
		Contributed Talk \hfill Melbourne, Australia\\
		{\bf Planets in Perculiar Places} \hfill {\bf April, 2018}\\
		Contributed Talk \hfill Sydney, Australia\\
		{\bf International Women's Day Science in the Pub} \hfill {\bf March, 2018}\\
		Invited Talk \hfill Canberra, Australia\\
		{\bf 12th ANITA Theory Workshop} \hfill {\bf February, 2018}\\
		Contributed Talk \hfill Perth, Australia\\
		{\bf Franco-Australian Astrobiology and Exoplanet School and Workshop} \hfill {\bf December, 2017}\\
		Contributed Talk \hfill Canberra, Australia\\
		{\bf Annual Scientific Meeting of the Astronomical Society of Australia} \hfill {\bf July, 2017}\\
		Contributed Talk \hfill Canberra, Australia\\
		{\bf 11th ANITA Theory Workshop} \hfill {\bf February, 2017}\\
		Contributed Talk \hfill Hobart, Australia\\
		{\bf Mt Stromlo Students Seminars} \hfill {\bf December, 2016}\\
		Contributed Talk (Awarded Best Theme Talk) \hfill Canberra, Australia\\
		{\bf 6th Australian Exoplanet Workshop} \hfill {\bf November, 2016}\\
		Contributed Talk \hfill Melbourne, Australia\\
		{\bf Star Formation} \hfill {\bf August, 2016}\\
		Computational Astrophysics splinter session (Invited) \hfill Exeter, UK
		{\bf Annual Scientific Meeting of the Astronomical Society of Australia} \hfill {\bf July, 2016}\\
		Contributed Talk \hfill Sydney, Australia\\
		{\bf 10th ANITA Theory Workshop} \hfill {\bf February, 2016}\\
		Contributed Talk \hfill Melbourne, Australia\\
		{\bf 5th Australian Exoplanet Workshop} \hfill {\bf November, 2015}\\
		Contributed Talk \hfill Sydney, Australia\\
		{\bf 9th ANITA Theory Workshop} \hfill {\bf February, 2015}\\
		Contributed Talk \hfill Canberra, Australia\\
		
		\section{\sc Awards and Honors}
		\begin{list2}
			\item 2021: Kvinder i Fysik (Danish Women in Physics) Prize 2021 Nominee
			\item 2020: European Union INTERACTIONS Fellowship
			\item 2017: Joan Duffield Research Supplementary Scholarship
			\item 2015: Australian Postgraduate Award
			\item 2013: Macquarie University Research Training Scholarship
			\item 2012: Vacation Scholarship (Macquarie University)
			\item 2011: Vacation Scholarship (Macquarie University)
		\end{list2}
		
		\section{\sc Teaching and mentoring experience}
		{\bf  Niels Bohr Institute Masters Students} \hfill {\bf August, 2021 - Present}\\
		I am co-supervising two Masters student. One student is working on producing synthetic observations from my simulations and the other is building a pipeline using machine learning to fit synthetics observations to real observations of young protostars.\\
		{\bf  Niels Bohr Institute Bachelors projects} \hfill {\bf February-April, 2021}\\
		I supervised three groups of students for their bachelors' projects where we modelled the interiors of exoplanets using polytropes.\\
		{\bf Computational Astrophysics} \hfill {\bf November, 2019, 2020}\\
		Gave post-graduate level lectures on computational astrophysics reviewing hydrodynamics and modelling shock waves.\\
		{\bf Mt Stromlo Observatory Summer Research} \hfill {\bf December, 2017 - February, 2018}\\
		Co-supervised Isabella Gerard (currently a graduate student at Monash University) on a research project on turbulent magnetic fields and star formation. I am co-author on the paper published from this project.\\
		{\bf Mt Stromlo Observatory Winter School} \hfill {\bf June-July, 2017}\\
		Advised undergraduate students Lara Cullinane and Patrick Armstrong (currently a graduate students at ANU), Joshua Ho and Lillian Guo in planning observations and writing telescope proposals.
		
		\section{\sc Computer Skills} 
		\begin{list2}
			\item Computing Languages: Python, Fortran and html
			\item Applications: \LaTeX , \texttt{yt}, simulation codes \texttt{RAMSES}, \texttt{FLASH} and Enzo, analysis of hdf5 files from hydrodynamic simulations, reducing observational data in fits files, retrieving radial velocities.  
			\item Operating Systems:  Unix/Linux, Windows, and Mac.
		\end{list2}
		
		\section{\sc Other Academic Services}
		\begin{list2}
			\item Reviewer for Monthly Notices of the Royal Astronomical Society
			\item Founded of Astronomy on Tap Copenhagen in 2020.
			\item Treasurer of Kvinder i Fysik (the Danish women in physics society) from 2019 to present.
			\item Contributed two popular science articles to the Sunday Space in the Canberra Times. 
			\item Member of the Local Organising Committee for the 2017 Harley Wood Winter School and Annual Scientific Meeting of the Astronomical Society of Australia.
			\item Member of the Science Organising Committee for the 2016 Harley Wood Winter School.
			\item Chair of the Organising Committee for the 2016 Mt Stromlo Student Seminars.
		\end{list2}
		
		%\vspace{0.3in}
		\section{\sc Referee Details}
		\begin{list2}
			%\item Professor Orsola De Marco, Department of Physics and Astronomy, Macquarie University, Sydney NSW 2109, Australia. tel: +61 2 9850 4241 , email: orsola.demarco@mq.edu.au
			%\item Associate Professor Michael Ireland, Research School of Astronomy and Astrophysics, Australian National University, Research School of Astronomy \& Astrophysics, Mount Stromlo Observatory, Cotter Road, Weston Creek, ACT 2611, Australia. tel: +61 2 6125 0288, email: michael.ireland@anu.edu.au
			\item Associate Professor Troels Haugb{\o}lle, Center for Star and Planet formation, University of Copenhagen, Geology Museum, Øster Voldgade 5-7, 1350 København K, tel: +45 35 32 11 41, email: haugboel@nbi.ku.dk
			\item Dr Christoph Federrath, Research School of Astronomy and Astrophysics, Australian National University, Research School of Astronomy \& Astrophysics, Mount Stromlo Observatory, Cotter Road, Weston Creek, ACT 2611, tel: +61 2 6125 0217, email: christoph.federrath@anu.edu.au
			\item Professor Jes Kristian J{\o}rgensen, Center for Star and Planet formation, University of Copenhagen, Geology Museum, Øster Voldgade 5-7, 1350 København K, tel:  +45 35 32 41 86, email: jeskj@nbi.ku.dk
			%\item Dr Brad Tucker, Research School of Astronomy and Astrophysics, Australian National University, Research School of Astronomy \& Astrophysics, Mount Stromlo Observatory, Cotter Road, Weston Creek, ACT 2611, tel: +61 2 6125 6711, email: brad.tucker@anu.edu.au 
		\end{list2}
	
		%\newpage
\section{\sc Refereed Publications}
\begin{list1}
	\item[]{\bf Kuruwita et al.}, \emph{The dependence of episodic accretion on eccentricity during the formation of binary stars}, 2020, Astronomy \& Astrophysics, 641, A59
	\begin{list2}
		\item Lead author, and conductor of research and analysis.\\
	\end{list2}
	\item[] {\bf Kuruwita \& Federrath}, \emph{The role of turbulence during the formation of circumbinary discs}, 2019, Monthly Notices of the Royal Astronomical Society, 486, 3647-3663
	\begin{list2}
		\item Lead author, and conductor of research and analysis.\\
	\end{list2}
	\item[] {\bf Kuruwita et al.}, \emph{Multiplicity of disc-bearing stars in Upper Scorpius and Upper Centaurus-Lupus}, 2018, Monthly Notices of the Royal Astronomical Society, 480, 5099–5112
	\begin{list2}
		\item Lead author, and conductor of research and analysis.
		\item Collected the majority of observations.\\
	\end{list2}
	\item[] {\bf Kuruwita et al.}, \emph{Binary star formation and the outflows from their discs}, 2017, Monthly Notices of the Royal Astronomical Society, 470, 1626-1641
	\begin{list2}
		\item Lead author, and conductor of research and analysis.\\
	\end{list2}
	\item[] {\bf Kuruwita et al.}, \emph{Considerations on the role of fall-back discs in the final stages of the common envelope binary interaction}, 2016, Monthly Notices of the Royal Astronomical Society, 461, 486-496
	\begin{list2}
		\item Lead author, and conductor of research and analysis.\\
	\end{list2}
	\item[]{J{\o}rgensen, J. \& {\bf Kuruwita, R.} et al}, \emph{Binarity of a protostar affects the evolution of the disk and planets}, 2021, Nature, \emph{Accepted}
	\begin{list2}
		\item Lead the theoretical component of paper. Conducted analysis of simulations used for comparison with observations.\\
	\end{list2}
	\item[] Gerrard et al., \emph{The role of magnetic field structure in the launching of protostellar jets}, 2019, Monthly Notices of the Royal Astronomical Society, 485, 5532-5542
	\begin{list2}
		\item Co-supervised Gerrard in running simulations and analysing them\\
	\end{list2}
	\item[] Green et al., \emph{Testing the binary trigger hypothesis in FUors}, 2016, The Astrophysical Journal, 830, 29
	\begin{list2}
		\item Obtained observational data with Keck and commented on paper drafts.\\
	\end{list2}
	\item[] Childress et al., \emph{The ANU WiFeS SuperNovA Programme (AWSNAP)}, 2016, Publications of the Astronomical Society of Australia, 33, 29
	\begin{list2}
		\item Obtained observational data with Australian National University 2.3m telescope.\\
	\end{list2}
	\item[] Little et al., \emph{Phase-stepping interferometry of GaAs nanowires: Determining nano-wire radius}, 2013, Applied Physical Letters, 103, 161107
	\begin{list2}
		\item Obtained experimental data with white light interferometry of nanowires.\\
	\end{list2}
\end{list1}
		
		%\newpage
		%\section{\sc Statement of Research Interests and Goals}
		
		%The importance of binary stars has been emphasised regularly with a significant portion of stars ($\sim50\%$) being born in binary or multiple star systems (Raghaven et al. (2010), Moe \& Stefano (2017)). Binary stars are also the progenitors to many phenomena such as Type 1a supernovae used for cosmology, gravitational waves and certain morphologies of planetary nebulae. Due to the significance that binaries have over various astrophysical processes, my research has mostly focused on the evolution of binary star systems.
		
		%My Masters of Research focused on the fall-back of bound material onto post-common envelope binaries. Previous simulations of the common envelope phase of binary stars found that the giant envelope is lifted during this phase, but remains mostly bound to the system. At the same time, the orbital separation is greatly reduced, but in most simulations it levels off at values larger than measured from observations (Passy et al. 2012). We conjectured that during the post-in-spiral phase the bound envelope gas would return to the system and provide another opportunity to further unbind the envelope and tighten the orbit. This work was carried out using 3D hydrodynamical simulations with the code Enzo and continued on from the work of Passy et al. (2012). We found that the simulated fall-back event reduces the orbital separation efficiently, but fails to unbind the gas before the separation levels off once again (Kuruwita et al. 2016). We also found that more massive fall-back disks reduce the orbital separation more efficiently, but the efficiency of unbinding remains invariably very low. From these results we deduced that unless a further energy source contributes to unbinding the envelope, all common envelope interactions would result in mergers. On the other hand, additional energy sources are unlikely to help, on their own, to reduce the orbital separation.
		
		%With the advent of the Kepler space telescope and the increasing number of known exoplanets my interest pivoted towards planets around binary star systems. To date we know of $\sim700$ planets in binary systems with $\sim20$ known planets in a circumbinary orbit. The majority of work on planet formation has focused on single stars. This is because humans have an anthropocentric bias towards single stars as our solar system has one star. However, with the increasing number of planets in binaries being discovered I wanted to investigate the environments in which planets in binaries could form. My PhD work focuses on binary star formation and dynamical evolution of protoplanetary disks in these systems. I have been conducting both theoretical and observational work on these areas of research. 
		
		%The theoretical focus of my PhD uses 3D ideal magneto-hydrodynamical simulations with the code FLASH to simulate binary star formation. I investigate how binarity affects the accretion of disk material, outflows, radiative feedback and dynamical evolution the disks. 
		
		%For my first project I simulated the formation of a single star, a tight binary star ($a\sim2.5$~AU) and a wide binary star ($a\sim45$~AU). We studied the outflows and jets from these systems to understand the contributions the circumstellar and circumbinary disks have on the efficiency and morphology of the outflows. In the single star and tight binary case, we obtained a single pair of jets launched from the system, while in the wide binary case two pairs of jets were observed (Kuruwita et al. 2017). This implied that in the tight binary case, the contribution of the circumbinary disk on the outflow is greater than that in the wide binary case. We also found that the single star case is the most efficient at transporting mass, linear and angular momentum from the system, while the wide binary case is less efficient. By studying the magnetic field structure, we deduced that the outflows in the single star and tight binary star case are magnetocentrifugally driven, whereas in the wide binary star case, the outflows are driven by a magnetic pressure gradient.
		
		%Further analysis of my wide binary simulations saw the disruption of circumstellar disks as the system passed periastron passage. This creates a hostile environment for planets to form. The next series of simulations investigate this disruption of circumstellar disks, but also introduces turbulence to the molecular cores that produce our binary stars. This is a natural progression in simulations as the universe is a turbulent place and plays a significant role in fragmentation and distribution of angular momentum. This work is finding that the inclusion of turbulence helps to build larger circumbinary disks, while also assisting with the ordering of magnetic fields through the disk for efficient launching of outflows (Kuruwita et al. 2018, \emph{Submitted} available at https://arxiv.org/abs/1810.10375).
		
		%The observational work of my PhD was performed to find the multiplicity of young disk-bearing objects in Upper Scorpius and Upper Centaurus-Lupus to determine the influence of multiplicity on disk persistence after $\sim5-20$~Myr. Disks were identified using infra-red (IR) excess from the Wide-field Infra-red Survey Explorer (WISE) survey. Our survey consisted of 55 US members and 28 UCL members, using spatial and kinematic information to assign a probability of membership (Rizzuto et al. 2015). Spectra were gathered from the ANU 2.3m telescope using the Wide Field Spectrograph (WiFeS) to detect radial velocity variations that indicate the presence of a companion. We identified 2 double-lined spectroscopic binaries, both of which have strong IR excess. We found the binary fraction of disk-bearing stars in US and UCL for periods up to 20 years to be $0.06^{+0.07}_{-0.02}$ and $0.13^{+0.06}_{-0.03}$ respectively (Kuruwita et al. 2018). Based on the multiplicity of field stars, we obtained an expected binary fraction of $0.12^{+0.02}_{-0.01}$. The determined binary fractions for disk-bearing stars does not vary significantly from the field, suggesting that the overall lifetime of disks may not differ between single and binary star systems. This may also suggest that the likelihood of planet formation around binary stars is similar to that of single stars.
		
		%At the American Museum of Natural History I would continue my numerical work on binary star formation and evolution. I am currently conducting research on episodic accretion during binary star formation and testing the binary trigger hypothesis. Episodic accretion caused by dynamical interactions has be investigated observationally (Green et al. (2016), Tofflemire et al. (2017)) and some theoretical work has been done with simple disk simulations (de Val-Borro et al, (2011), Gomez de Castro et al, (2013)). My simulations are some of the first showing episodic accretion correlated with the period of the formed binaries beginning at the collapse of the molecular core. I would like to investigate this episodic accretion on outflows and feedback during star formation. This work will also contribute to understanding radiation feedback during binary star formation. In the past year I have been incorporating a radiation module (Buntemeyer et al. 2014) into FLASH to understand how binarity affects radiation feedback and disk evolution, and accretion luminosity will play a significant role.
		
		%I would also like to conduct research into determining the separation ranges in which circumbinary planets are likely to form. This will help constraint targeted observations searching for these types of planets. Dynamical interactions and the ejection of a third companion is believed to be a common pathway for creating short period binaries (Armitage \& Clarke (1997), Reipurth (2000)). The ejection of this third companion may disturb circumbinary disks from which planets would form. Long period binaries are also more likely to host circumstellar disks than circumbinary disks, therefore the formation of circumbinary planets would be reduced around wider binaries. I would like to conduct a suite of simulations to assist the search for circumbinary planets by determining the type of binaries that are likely to form such planets.
		
		%Also along the lines of short period binaries forming via the ejection of a third companion, I would also like to investigate the survivability of disks during such dynamical interactions. For example, the systems AK Sco and DQ Tau are two very short period binary system ($\sim13$ and $\sim16$~days respectively (G{\"u}nther \& Kley, 2002)). These systems also host circumbinary disks. If these systems formed via the ejection of a third companion did these circumbinary disks survive the ejection or form at later stages?
		
		%My research interests cover a wide range from the birth of binary stars, to the evolution of disks and the formation of planets in these systems. If successful in this application, I believe I have the skills and background to integrate into the research teams at the American Museum of Natural History. My experience with numerical simulations, especially magneto-hydrodynamical codes, will contribute to rapidly becoming an active and efficient team member.
		
	\end{resume}
	
\end{document}









